% python classes slides - control statements
% (c) 2012 Kostiantyn Danylov aka koder 
% koder.mail@gmail.com
% distributed under CC-BY licence
% http://creativecommons.org/licenses/by/3.0/deed.en

\documentclass{article}
% XeLaTeX
\usepackage{xltxtra}
\usepackage{xunicode}
\usepackage{listings}
\usepackage[landscape]{geometry}

% Fonts
\setmainfont{DejaVu Sans} %{Arial}
\newfontfamily\cyrillicfont{Nimbus Roman No9 L} %{Arial}
\setmonofont{Courier New}
%\setmonofont{Ubuntu Mono}

%\setmonofont{DejaVu Sans Mono}

% Lang
\usepackage{polyglossia}
\setmainlanguage{russian}
\setotherlanguage{english}
\usepackage[dvipsnames,table]{xcolor}


\ifx\pdfoutput\undefined
\usepackage{graphicx}
\else
\usepackage[pdftex]{graphicx}
\fi

\lstset{
	language=python,
	keywordstyle=\color{Emerald},%\texttt, 
	commentstyle=\color{OliveGreen},%\texttt,
	stringstyle=\color{Bittersweet},%\texttt,
	tabsize=4,
	numbers=left,
	xleftmargin=10pt,
	morekeywords={with,as},	
	numberstyle=\large,
	%identifierstyle=\texttt,
	%basicstyle=\texttt,
}

\usepackage{hyperref}

\hypersetup{
	colorlinks=true,
	urlcolor=blue
}

\usepackage{float}
%\floatstyle{boxed} 
%\restylefloat{figure}
\usepackage[normalem]{ulem}

\input{files/python_cmds}
\begin{document}
\LARGE

%-------------------------------------------------------------------------------
\center{Блоки кода}
\begin{itemize}
	\item Блоки ограничивают участок кода, принадлежащий управляющей конструкции
	\item Начинаются с “:”, которым оканчивается конструкция 
	\item Все строки блока имеют уровень отступа равным начальной строке блока
	\item Отступы делаются с помошью табуляции или пробелов
	\item Блоки могут содержать другие блоки (с более глубокими отступами)
\end{itemize}
\vspace{15pt}
\begin{lstlisting}
	Some_contruction:
		y = 2
		z = x + y
	#end_of_block
\end{lstlisting}
\newpage

%-------------------------------------------------------------------------------
\center{Блоки кода}
\begin{itemize}
	\item Блоки это не области видимости переменных. Переменные видны и после выхода из блока
	\item \lstinline$pass$ – пустой блок
\end{itemize}
\newpage

%-------------------------------------------------------------------------------
\center{if - Условное выполнение участков кода}
\vspace{15pt}
\begin{lstlisting}
	if  condition1 :
	    pass # excuted if condition1 is true
	elif condition2 :
	    pass # excuted if condition1 is false and condition2 is true
	#... 
	else:
	    pass # executed if all conditions is false 
\end{lstlisting}
\newpage

%-------------------------------------------------------------------------------
\center{if}
\vspace{15pt}
\begin{lstlisting}
	x = 12
	sign = 0
	if x > 0:
	    print x, "positive"
	    sign = 1
	elif x < 0:
	    print x, "negative"
	    sign = -1
	else:
	    print x, "== 0"
	    sign = 0
\end{lstlisting}
\newpage

%-------------------------------------------------------------------------------
\center{inline if}
\vspace{15pt}
\begin{lstlisting}
	res = x if x >= 0 else -x
	# res = (x >= 0 ? x : -x)
\end{lstlisting}
\newpage

%-------------------------------------------------------------------------------
\center{while}
\vspace{15pt}
\begin{lstlisting}
	while condition:
		pass # executed while condition is true
	else:
		pass # if no error or break in body

	x = 1
	while x < 100:
		print x, "less than 100"
		x *= 2
\end{lstlisting}
\newpage

%-------------------------------------------------------------------------------
\center{for - цикл по множеству}
\vspace{15pt}
\begin{lstlisting}
	for x in iterable:
		func(x) # for each element in iterable
	else:
		pass # if no error or break in body

	sum = 0
	for x in range(100):
		sum += x
	print x  # 99 * 100 / 2

	for i in range(n): # xrange(n)
	    pass

	n = 121213

    dividers = []
    while n > 3:
        for divider in range(2, int(n ** 0.5) + 1):
            if n % divider == 0:
                break
        else:
            break
        n //= divider
        dividers.append(divider)

    if n != 1:
    	dividers.append(n)
\end{lstlisting}
\newpage

%-------------------------------------------------------------------------------
\center{for undercover}
\vspace{15pt}
\begin{lstlisting}
	for x in container:
	    f(x)

	# some times equal to

	_tmp = 0
	while _tmp < len(container):
	    x = container[_tmp]
	    f(x)
	    _tmp += 1
\end{lstlisting}
\newpage

%-------------------------------------------------------------------------------
\center{break \& сontinue как всегда}
\begin{itemize}
	\item \lstinline!break!  выходит из цикла
	\item \lstinline!continue! переходит к следующей итерации
\end{itemize}
\newpage

%-------------------------------------------------------------------------------
\center{Нет}
\begin{itemize}
	\item goto 
	\item switch + case 
	\item until 
	\item do{}while, do{}until
\end{itemize}
\newpage

%-------------------------------------------------------------------------------
\center{with}
\vspace{15pt}
\begin{lstlisting}
	with expression as var:
		block

	# mostly the same as
	
	var = expression
	var.__enter__()
	
	block
	
	if error_happened:
		if var.__exit__(error_data):
			# pass_error_further
		else:
			# supress_error
	else:
		var.__exit__()
\end{lstlisting}
\newpage

%-------------------------------------------------------------------------------
\center{использование with}
\vspace{15pt}
\begin{lstlisting}
	with open(r“C:\xxx.bin”, "w") as fd:
	    fd.write(“-” * 100 + "\n")
	    fd.write(“+” * 100 + "\n")

	with open(r“C:\xxx.bin”, "r") as fd:
	    for line in fd:
	    	print line

	with db.cursor() as cur:
	    curr.execute(update_request_1)
	    curr.execute(update_request_2)
		# commit or rollback
\end{lstlisting}
\newpage

%-------------------------------------------------------------------------------
\center{List comprehension}
\vspace{15pt}
\begin{lstlisting}
	res = [func(i) for i in some_iter if func2(i)]

	res = ["{:.2f}".format(i ** 0.5) 
				for i in [-1, 0, 1, 2, 3] 
					if i >= 0]
	
	res == ['0.00', '1.00', '1.41', '1.73']

	res = [(i + 0j) ** 0.5 for i in [-1, 0 ,1, 2, 3]]
	res = {func(i) for i in some_iter if func2(i)}
\end{lstlisting}
\newpage

%-------------------------------------------------------------------------------
\center{Функции - минимум}
\begin{lstlisting}
	def func_name1(param1, param2):
		"documentation"
		# block
		x = param1 + param2
		return x

	def func_name2(param1, param2):
		"documentation"
		# block
		x = param1 + param2
		if x > 0:
			return x
		else:
			return 0
\end{lstlisting}
\newpage

%-------------------------------------------------------------------------------
\center{Unit tests - find}
\begin{lstlisting}
	assert find("abc", "b") == 1
	assert find("abc", "b") == "abc".find("b")

	assert find("abc", "a") == 0
	assert find("abca", "a") == 0
	assert find("dabca", "a") == 1
	assert find("", "a") == -1
	assert find("a", "a") == 0
	assert find("ab", "abc") == 0
	assert find("b" * 1000 + "abc", "abc") == 1000
	assert find("b" * 1000 + "abc", "abcd") == -1

	all_symbols = "".join([chr(i) for i in range(255)])
	assert find(all_symbols, chr(100)) == 100

	assert find("", "") == 0
	assert find("", "") == "".find("")
\end{lstlisting}
\newpage
%-------------------------------------------------------------------------------
\center{Program template}
\begin{lstlisting}
	#!/usr/bin/end python
	# -*- coding:utf8 -*-
	......

	def main():
		res = 0
		.....
		return res

	if __name__ == "__main__":
		exit(main())
\end{lstlisting}
\newpage

%-------------------------------------------------------------------------------
\center{ДЗ}
\begin{itemize}
	\item Написать строковые функции xfind, xreplace, xsplit, xjoin используя срезы строк 
		    (без применения других методов строк). \\
		  xfind(s1, s2) == s1.find(s2) \\
		  xreplace(s1, s2, s3) == s1.replace(s2, s3) \\
		  xsplit(s1, s2) == s1.split(s2) \\
		  xjoin(s, array) == s.join(array)

	\item Написать кодирование и декодирование файла по Хаффману. 
		  На диске есть файл с именем "input.txt". 
		  Его нужно прочитать, закодировать символы использую алгоритм Хаффмана
		  и записать результат в output.bin. В решении должно быть две функции
		  hf\_encode(string) str->str, и hf\_decode(string) str->str.
		  Первая кодирует, вторая декодирует. Входными элементами для алгоритма
		  являются отдельные байты файла.

	\item Написать интерпретатор подмножества языка forth. 

			Программа на Forth состоит из набора команд(слов),
			некоторые из которых имеют параметры. Для хранения данных используется стек -
			команды получают свои операнды с вершины стека и туда же сохраняют результаты.
			В  подмножестве 5 команд: \\

			put значение - ложит значение на вершину стека.
			Значение может быть числом или строкой. 
			Строка заключается в кавычки, внутри строки кавычек быть не может \\

			pop - убирает значение с вершины стека \\
			add - изымает из стека 2 значения, складывает их, кладет результат в стек\\
			sub - изымает из стека 2 значения, вычитает их, кладет результат в стек \\
			print - вынимает из стека 1 значение, печатает его. \\


\begin{verbatim}
	put 3
	put "asdaadasdas"
\end{verbatim}

			Каждая команда начинается с новой строки. Строки, начинающиеся с '\#' - комментарии.
			Ваша программа должна содержать функцию eval\_forth(), принимающую строку на языке
			forth и исполняющую ее. По умолчанию из main вызывать eval\_forth("example.frt")
			Пример, если в example.rft будет:

\begin{verbatim}
	put 1
	put 3
	add
	print
\end{verbatim}

			То программа должна напечатать '4'. Сложение имеет такой же смысл, как и в питоне. 
			Вычитание для строк не определено, все входные данные проверять с помощью assert.

\end{itemize}
\newpage

%-------------------------------------------------------------------------------
\end{document}
