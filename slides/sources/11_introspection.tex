% python classes slides - profiling and optimization
% (c) 2014 Kostiantyn Danylov aka koder 
% koder.mail@gmail.com
% distributed under CC-BY licence
% http://creativecommons.org/licenses/by/3.0/deed.en

\documentclass{article}
% XeLaTeX
\usepackage{xltxtra}
\usepackage{xunicode}
\usepackage{listings}
\usepackage[landscape]{geometry}

% Fonts
\setmainfont{DejaVu Sans} %{Arial}
\newfontfamily\cyrillicfont{Nimbus Roman No9 L} %{Arial}
\setmonofont{Courier New}
%\setmonofont{Ubuntu Mono}

%\setmonofont{DejaVu Sans Mono}

% Lang
\usepackage{polyglossia}
\setmainlanguage{russian}
\setotherlanguage{english}
\usepackage[dvipsnames,table]{xcolor}


\ifx\pdfoutput\undefined
\usepackage{graphicx}
\else
\usepackage[pdftex]{graphicx}
\fi

\lstset{
	language=python,
	keywordstyle=\color{Emerald},%\texttt, 
	commentstyle=\color{OliveGreen},%\texttt,
	stringstyle=\color{Bittersweet},%\texttt,
	tabsize=4,
	numbers=left,
	xleftmargin=10pt,
	morekeywords={with,as},	
	numberstyle=\large,
	%identifierstyle=\texttt,
	%basicstyle=\texttt,
}

\usepackage{hyperref}

\hypersetup{
	colorlinks=true,
	urlcolor=blue
}

\usepackage{float}
%\floatstyle{boxed} 
%\restylefloat{figure}
\usepackage[normalem]{ulem}

\input{files/python_cmds}
\begin{document}
\LARGE

%-------------------------------------------------------------------------------
\begin{center} Интроспекция \end{center}
\begin{itemize}
    \item Добраться можно до всего чего угодно (почти) и бОльшую часть этого можно поменять \"на лету\"
    \item DRY
\end{itemize}
\newpage

%----------------------------------------------------------------------------------------------------------------------------
\begin{center} Внутри объектов \end{center}
\begin{lstlisting}
class T(object):
    val = 1
    def __init__(self, data):
        self.data = data
        self.__x = 1

    def func(self):
        return self.data
t = T(2)
t.__dict__ == {'_T__x': 1, 'data': 1}
t.__class__ is T
\end{lstlisting}
\newpage

%----------------------------------------------------------------------------------------------------------------------------
\begin{center} Внутри объектов \end{center}
\begin{lstlisting}
x = t.func
x() == 2

y = T.func
y(x) == 2

x2 = y.__get__(t)
x2() == 2
\end{lstlisting}
\newpage

%----------------------------------------------------------------------------------------------------------------------------
\begin{center} Внутри методов \end{center}
\begin{lstlisting}
x = t.func
x() == 2
x.im_func, x.im_self, x.im_class

y = T.func
y(x) == 2

x2 = y.__get__(t)
x2() == 2
\end{lstlisting}
\newpage

%----------------------------------------------------------------------------------------------------------------------------
\begin{center} Внутри классов \end{center}
\begin{lstlisting}
T.__dict__ == {'__dict__': <attribute '__dict__' of 'T' objects>,
               '__doc__': None,
               '__init__': <function __main__.__init__>,
               '__module__': '__main__',
               '__weakref__': <attribute '__weakref__' of 'T' objects>,
               'func': <function __main__.func>,
               'val': 1}
\end{lstlisting}
\newpage

%----------------------------------------------------------------------------------------------------------------------------
\begin{center} Внутри функций \end{center}
\begin{lstlisting}
def test(v1, v2=12):
    return v1 + v2

test.func_code # code object
test.func_defaults == (12, )

test.x = 0
test.__dict__ == {'x': 0}

#func_closure, func_doc, func_globals, func_name
\end{lstlisting}
\newpage

%----------------------------------------------------------------------------------------------------------------------------
\begin{center} Внутри функций \end{center}
\begin{lstlisting}
import dis
import inspect

dis.dis(test)
#  2 0 LOAD_FAST     0 (v1)
#    3 LOAD_FAST     1 (v2)
#    6 BINARY_ADD       
#    7 RETURN_VALUE

inspect.getargspec(test)
# ArgSpec(args=['v1', 'v2'], varargs=None, keywords=None, defaults=(12,))
\end{lstlisting}
\newpage

%----------------------------------------------------------------------------------------------------------------------------
\begin{center} Внутри функций \end{center}
\begin{lstlisting}
import ast
print ast.dump(ast.parse("x = 1 + r"))

Module(body=[
    Assign(
        targets=[Name(id='x', ctx=Store())], 
        value=BinOp(
                left=Num(n=1), 
                op=Add(), 
                right=Name(id='r', ctx=Load())
            )
        )
    ])
\end{lstlisting}
\newpage

%----------------------------------------------------------------------------------------------------------------------------
\begin{center}Пример\end{center}
Нужно сгенерировать вагончик API к имеющемуся коду.
СLI, HTTP, RPC, whatever.
\newpage

%----------------------------------------------------------------------------------------------------------------------------
\begin{center}Пример\end{center}
\begin{lstlisting}

class Command(objects):
    class Params:
        pass

def create_cli(cmd):
    pass

@classmethod
def validate(self, **params):
    pass
\end{lstlisting}
\newpage

%----------------------------------------------------------------------------------------------------------------------------
\end{document}
