% python classes slides - introduction
% (c) 2012 Kostiantyn Danylov aka koder 
% koder.mail@gmail.com
% distributed under CC-BY licence
% http://creativecommons.org/licenses/by/3.0/deed.en

\documentclass{article}
% XeLaTeX
\usepackage{xltxtra}
\usepackage{xunicode}
\usepackage{listings}
\usepackage[landscape]{geometry}

% Fonts
\setmainfont{DejaVu Sans} %{Arial}
\newfontfamily\cyrillicfont{Nimbus Roman No9 L} %{Arial}
\setmonofont{Courier New}
%\setmonofont{Ubuntu Mono}

%\setmonofont{DejaVu Sans Mono}

% Lang
\usepackage{polyglossia}
\setmainlanguage{russian}
\setotherlanguage{english}
\usepackage[dvipsnames,table]{xcolor}


\ifx\pdfoutput\undefined
\usepackage{graphicx}
\else
\usepackage[pdftex]{graphicx}
\fi

\lstset{
	language=python,
	keywordstyle=\color{Emerald},%\texttt, 
	commentstyle=\color{OliveGreen},%\texttt,
	stringstyle=\color{Bittersweet},%\texttt,
	tabsize=4,
	numbers=left,
	xleftmargin=10pt,
	morekeywords={with,as},	
	numberstyle=\large,
	%identifierstyle=\texttt,
	%basicstyle=\texttt,
}

\usepackage{hyperref}

\hypersetup{
	colorlinks=true,
	urlcolor=blue
}

\usepackage{float}
%\floatstyle{boxed} 
%\restylefloat{figure}
\usepackage[normalem]{ulem}

\input{files/python_cmds}

\begin{document}
\LARGE

%-------------------------------------------------------------------------------
\begin{center} \includegraphics[]{images/python-logo-master-v3-TM-flattened.png} \end{center}
\begin{itemize}
    \item Данилов Константин
    \item koder.mail@gmail.com - тему начинайте с "PythonClasses:XXX"
    \item Слайды и доп материалы - \\
            \href{https://github.com/koder-ua/python-classes/tree/master/slides/pdf}{github/koder-ua}
\end{itemize}
\newpage

%-------------------------------------------------------------------------------
\begin{center} Кто это \end{center}
\begin{center} \includegraphics{images/mp_crazy.jpg} \end{center}
\newpage

%-------------------------------------------------------------------------------
\begin{center} \Large{Monty Python's flying circus} \end{center}
\begin{center} \includegraphics{images/montypython.jpg} \end{center}
\newpage

%-------------------------------------------------------------------------------
\begin{center} Почему Python (-) \end{center}
\begin{itemize}
    \item Python достаточно сложен для изучения, особенно если разбираться во всех тонкостях
    \item Невозможность качественного статического анализа (до 3.5)
    \item Слабая поддержка инструментальными средствами (по сравнению с С++/С\#/Java)
    \item Несколько десятков способов официальных способов выстрелить себе в ногу (и попасть в голову соседу)
    \item Низкая скорость исполнения (по сравнению с компилируемыми языками)
    \item Высокая сложность написания JIT (по сравнению с Javascript)
    \item Поддержку многих высокоуровневых возможностей сложно сделать эффективной 
          == она скорее всего всегда будет медленной
            (перегрузка функций, поиск по образцу, оптимизацию хвостовой рекурсии)
    \item В дизайне языка допускаются ошибки
\end{itemize}
\newpage

%-------------------------------------------------------------------------------
\begin{center} Почему Python (+) \end{center}
\begin{itemize}
    \item ~Второй по лаконичности язык (уступает Perl) при этом имея строгий 
            минималистичный синтаксис. Требует значительно меньше кода, чем компилируемые языки.
    \item Ядро языка чрезвычайно компактно. Нет декларируемых конструкций. 
            Большая часть конструкция - синтаксический сахар
    \item В несколько десятков строк можно реализовать большинство отсутствующих языковых возможностей
           (но скорость...)
    \item Все это + интроспекция = pythonic библиотеки 
        (документация по библиотеке умещается на заглавной странице сайта)
    \item Батарейки в комплекте
    \item Переносимость, обратная совместимость, прогнозируемая и формализованная модель разработки
\end{itemize}
\newpage

%-------------------------------------------------------------------------------
\begin{center} Python \end{center}
\begin{itemize}
    \item Язык программирования ориентированный на скорость работы программиста и
        быстрое освоение библиотек/легкое использование
    \item Мультипарадигменный - процедурный, ООП, функциональный. Позволяет легко реализовывать другие парадигмы
    \item Opensource - https://github.com/python/cpython
    \item С-подобный минималистичный (учитывая уровень языка) строгий синтаксис
    \item Разрабатывается с конца 80х. В 2001 выходит v2.1 и создается PSF.
    \item Развивается открытым сообществом под руководством Гвидо Ван Россума
    \item Роль стандарта выполняет \href{http://www.python.org}{CPython}
\end{itemize}
\newpage

%-------------------------------------------------------------------------------
\begin{center} Распространение \end{center}
\begin{itemize}
    \item ~1-8й язык по популярности
    \item Научные расчеты и постобработка данных
    \item Web
    \item GUI, Системы управления, встраиваемый язык
    \item Склеивание компонентов, написанных на С/С++
    \item Xen, apt, mercurial, Trac, youtube, GAE, .....
\end{itemize}
\newpage

%-------------------------------------------------------------------------------
\begin{center} Версии и реализации \end{center}
\begin{itemize}
    \item Две ветви 2.X(2.7.14) и 3.X(3.6.3)/3.7.0a3, 3.7.0 запланирован на 15 Июля 2018
    \item 2.8 (наверное) не будет
    \item Внутри каждой ветви поддерживается полная обратная совместимость (для py файлов)
    \item 3.X (Python 3k) достаточно близка к 2.Х, содержит несовместимые исправления 
    		архитектурных ошибок, внесенных в язык на ранних стадиях
    \item print стал функцией, переработка юникод подсистемы, ввода-вывода и др
    \item Тем не менее любая нетривиальная программа на 2.X должна быть изменена для запуска на 3.X
    \item Все реализации в значительной мере - интерпретаторы
    \item PyPy, Jython, IronPython, Stackless Python, Py4A, QPython  ....
    \item Работает на всех распространенных платфрмах - Intel win/lin/mac/bsd/.., Sun, Power, 
    		ARM(Android), Symbian, .....
\end{itemize}
\newpage

%-------------------------------------------------------------------------------
\begin{center} Процесс разработки языка \end{center}
\begin{itemize}
    \item Формализованный и бюрократический подход к изменениям в языке
    \item Новые версии каждые ~1.5 года
    \item Разработка ведется через python-dev \& python-ideas списки рассылки
    \item Все рассылки открытые
    \item Все изменения и предложения описаны в PEP's (Python Enhancement Proposals) 
            в т.ч. и отклоненные
    \item python-ideas -> python-dev -> PEP XXXX -> ... -> {accepted/rejected} by Guido
\end{itemize}
\newpage

%-------------------------------------------------------------------------------
\begin{center} Библиотеки \end{center}
\begin{itemize}
    \item Очень широкий спектр библиотек
    \item Web, Сети, DB, Визуализация, Научные расчеты, XML, GUI,...
    \item Есть привязки почти для всех крупных C/C++ библиотек
    \item Cython, SWIG, SIP,...
\end{itemize}
\newpage

%-------------------------------------------------------------------------------
\begin{center} Установка windows \end{center}
\begin{itemize}
    \item Windows: 2.7 с http://www.python.org/download/
    или ActivePython c http://www.activestate.com/activepython/downloads
    \item Почти все библиотеки - http://pypi.python.org/pypi
    \item Ручная установка библиотеки - exe или распаковать zip, и python setup.py install
    \item Пакетные менеджеры - pip, setuptools \\
    	pip install имя\_пакета==версия или easy\_install имя\_пакета==версия
    \item virtualenv - создание изолированных окружений
    \item C:{\textbackslash}Python2.7{\textbackslash}lib{\textbackslash}site-packages
\end{itemize}
\newpage

%-------------------------------------------------------------------------------
\begin{center} Установка linux \end{center}
\begin{itemize}
    \item Linux: apt-get install python python-setuptools python-pip python-virtualenv
    \item Почти все библиотеки - http://pypi.python.org/pypi
    \item Ручная установка библиотеки - exe или распаковать zip, и python setup.py install
    \item Пакетные менеджеры - pip, setuptools \\
    	    pip install имя\_пакета==версия или easy\_install имя\_пакета==версия
    \item virtualenv - создание изолированных окружений (с 3.3 включен в stdlib)
    \item \href{https://github.com/yyuu/pyenv}{pyev} менеджер версий
    \item /usr/lib/python2.7/dist-packages
    \item vagrant/docker
\end{itemize}
\newpage

%-------------------------------------------------------------------------------
\begin{center} IDE \& Co \end{center}
\begin{itemize}
    \item Eclipse + pydev, PyCharm, PyScripter, Python for VS, ...
    \item Sublime Text, Notepad++, Texmate, Vim, Emacs,....
    \item PEP8, pylint >= pychecker >= pyflakes
    \item \href{http://wiki.python.org/moin/IntegratedDevelopmentEnvironments}{python IDE list}
\end{itemize}
\newpage

%-------------------------------------------------------------------------------
\begin{center} Другие инструментальные средства \end{center}
\begin{itemize}
    \item \href{https://www.pythonanywhere.com/}{www.pythonanywhere.com} интерпретатор питона с web интерфейсом
    \item \href{http://www.codeskulptor.org/}{www.codeskulptor.org} еще один
    \item \href{http://www.pythontutor.com/visualize.html}{pythontutor} Визуализация исполнения python программ 
\end{itemize}
\newpage

%-------------------------------------------------------------------------------
\begin{center} Отладчики \end{center}
\begin{itemize}
    \item pdb
    \item rpdb2 + winpdb (требует wxpython)
    \item pudb
    \item pydb
\end{itemize}
\newpage

%-------------------------------------------------------------------------------
\begin{center} Интерпретатор \end{center}
\begin{itemize}
    \item python
    \item ipython (pyreadline, http://ipython.org/pyreadline.html)
    \item ipython notebook (pyzmq + tornado)
    \item ipython qtconsole (PyQt4)
    \item bpython
\end{itemize}
\newpage

%-------------------------------------------------------------------------------
\begin{center} Обзор. Программа на Python \end{center}
\begin{itemize}
    \item Набор файлов на Python
    \item Каждый файл рассматривается как набор строк
\end{itemize}
\begin{center} Заголовок программы \end{center}
\vspace{15pt}
\begin{lstlisting}
    #!/usr/bin/env python
    # -*- coding:utf8 -*-
\end{lstlisting}

\begin{itemize}
    \item Часть строки после \# - комментарий
    \item Длинные строки можно переносить, поставив в конце 
    		"\textbackslash". После него не должно идти пробелов
\end{itemize}
\newpage

%-------------------------------------------------------------------------------
\begin{center} Пример программы \end{center}
\vspace{15pt}
\begin{lstlisting}
    #!/usr/bin/env python
    # -*- coding:utf8 -*-
    x = 1
    print "Hello, world!"
    print "x =", x
    print "This is definitelly " + \
            "too long line"
\end{lstlisting}
\newpage

%-------------------------------------------------------------------------------
\begin{center} Исполнение программы \end{center}
\begin{itemize}
    \item При первой загрузке программа компилируется в байтокод для 
    	виртуального стекового процессора, встроенного в CPython
    \item .py -> .pyc (python compiled)
    \item python -o  .py -> .pyo. Удаление assert, etc
    \item .pyd(dll), .so - бинарные модули
\end{itemize}
\newpage

%-------------------------------------------------------------------------------
\begin{center} Print \end{center}
\begin{itemize}
    \item Вывод набора значений или переменных на экран
    \item Автоматически вставляет пробелы между значениями и перенос строки в конце
\end{itemize}
\vspace{15pt}
\begin{lstlisting}
    print(var1, var2)
    print(1, 2, "34")
    print(x, y, x + y, end="") # no new line
\end{lstlisting}
\newpage

%-------------------------------------------------------------------------------
\begin{center} Внешние модули \end{center}
\begin{itemize}
    \item Библиотеки на Python называются модулями или пакетами
    \item \lstinline$import module$ подключает модуль "module" в программу. 
    После этого его элементы доступны как "module.name"
    \item \lstinline$from module import *$ напрямую включает все элементы "module" в программу. 
    \item \lstinline$from module import xxx,yyy$
        напрямую включает выбранные элементы "module" в программу. 
\vspace{15pt}

\begin{lstlisting}
    import os
    from os import listdir
    from os import *

    print os.listdir(".")
\end{lstlisting}
\end{itemize}
\newpage

%-------------------------------------------------------------------------------
\begin{center} Ошибки \end{center}
\begin{itemize}
    \item При возникновении ошибки python порождает исключение, 
    			передающееся вверх по стеку до первого обработчика.
    \item Если в программе не определен ни один обработчик ошибок этого типа, то исключение
    			передается в обработчик по умолчанию, печатающий информацию о исключении
    			и завершающий программу.
\end{itemize}

{
\Large
\begin{lstlisting}
    def f1(a, b):
        return a / b

    def f2(m):
        return f1(2, m)

    f2(0)
\end{lstlisting}

\begin{lstlisting}
    Traceback (most recent call last):
      File "/tmp/m.py", line 7, in <module>
        f2(0)
      File "/tmp/m.py", line 5, in f2
        return f1(2, m)
      File "/tmp/m.py", line 2, in f1
        return a / b
    ZeroDivisionError: integer division or modulo by zero
\end{lstlisting}
}

\newpage

%-------------------------------------------------------------------------------
\begin{center} Справка и исследование объектов \end{center}
\begin{itemize}
    \item help(obj)
    \item obj?    -- help
    \item obj??   -- help + source
    \item obj.<tab>  -- extension
    \item dir(obj)
\end{itemize}

\begin{lstlisting}
    In [1]: import antigravity

    In [2]: antigravity??
    Type:       module
    String Form:<module 'antigravity' from '...\lib\antigravity.py'>
    File:       c:\dev\python\python27_x86\lib\antigravity.py
    Source:

    import webbrowser

    webbrowser.open("http://xkcd.com/353/")
\end{lstlisting}
\newpage

%-------------------------------------------------------------------------------
\end{document}
