% (c) 2017 Kostiantyn Danylov aka koder 
% koder.mail@gmail.com
% distributed under CC-BY licence
% http://creativecommons.org/licenses/by/3.0/deed.en

\documentclass{article}
% XeLaTeX
\usepackage{xltxtra}
\usepackage{xunicode}
\usepackage{listings}
\usepackage[landscape]{geometry}

% Fonts
\setmainfont{DejaVu Sans} %{Arial}
\newfontfamily\cyrillicfont{Nimbus Roman No9 L} %{Arial}
\setmonofont{Courier New}
%\setmonofont{Ubuntu Mono}

%\setmonofont{DejaVu Sans Mono}

% Lang
\usepackage{polyglossia}
\setmainlanguage{russian}
\setotherlanguage{english}
\usepackage[dvipsnames,table]{xcolor}


\ifx\pdfoutput\undefined
\usepackage{graphicx}
\else
\usepackage[pdftex]{graphicx}
\fi

\lstset{
	language=python,
	keywordstyle=\color{Emerald},%\texttt, 
	commentstyle=\color{OliveGreen},%\texttt,
	stringstyle=\color{Bittersweet},%\texttt,
	tabsize=4,
	numbers=left,
	xleftmargin=10pt,
	morekeywords={with,as},	
	numberstyle=\large,
	%identifierstyle=\texttt,
	%basicstyle=\texttt,
}

\usepackage{hyperref}

\hypersetup{
	colorlinks=true,
	urlcolor=blue
}

\usepackage{float}
%\floatstyle{boxed} 
%\restylefloat{figure}
\usepackage[normalem]{ulem}

\input{files/python_cmds}
\begin{document}
\LARGE

%-------------------------------------------------------------------------------
\begin{center} Paramiko sftp \end{center}
\vspace{15pt}
\begin{lstlisting}
    t = paramiko.Transport((hostname, port))
    try:
        t.connect(username=username, password=password)
        sftp = paramiko.SFTPClient.from_transport(t)
        sftp.get(source, dest)
    finally:
        t.close()
\end{lstlisting}
\newpage

%-------------------------------------------------------------------------------
\begin{center} Paramiko ssh \end{center}
\vspace{15pt}
\begin{lstlisting}
    client = paramiko.SSHClient()
    client.load_system_host_keys()
    client.set_missing_host_key_policy(paramiko.WarningPolicy)
    client.connect(host, username=username,
                   pkey=paramiko.RSAKey.from_private_key_file(key_file, password=SSH_KEY_PASSWD),
                   look_for_keys=False)
    client.connect(host, username=username, password=password)
    client.exec_command(command)
\end{lstlisting}
\newpage

%-------------------------------------------------------------------------------
\begin{center} HyperV \end{center}
\begin{itemize}
    \item WMI
    \item pywinrm
    \item IP
    \item http://antigluk.blogspot.com/2014/04/control-hyper-v-with-python.html
\end{itemize}
\newpage

%-------------------------------------------------------------------------------
\begin{center} WMI \end{center}
\begin{itemize}
    \item https://pypi.python.org/pypi/WMI/
    \item http://timgolden.me.uk/python/wmi/tutorial.html
\end{itemize}
\newpage

%-------------------------------------------------------------------------------
\begin{center} WinRM \end{center}
\begin{lstlisting}
import winrm
s = winrm.Session('windows-host.example.com', auth=('john.smith', 'secret'))
r = s.run_cmd('ipconfig', ['/all'])
print(r.status_code)
print(r.std_out)
r = s.run_ps(ps_script)
\end{lstlisting}
\newpage

%-------------------------------------------------------------------------------
\begin{center} Ansible installation \end{center}
\begin{lstlisting}
# python3 -m venv ansible
# source ansible/bin/activate
# pip install 'pywinrm>=0.2.2'
# pip install ansible
# ansible --version
\end{lstlisting}
\newpage

%-------------------------------------------------------------------------------
\begin{center} Ansible \end{center}
\begin{itemize}
    \item Inventory
    \item Config
    \item Playbooks
\end{itemize}
\begin{verbatim}

- hosts
- xxx1.yaml
- xxx2.yaml
- group_vars/
    - HOST_GROUP_NAME1.yml
    - HOST_GROUP_NAME2.yml
    - all.yml

\end{verbatim}
\newpage

%-------------------------------------------------------------------------------
\begin{center} Inventory \end{center}
\vspace{15pt}
\begin{lstlisting}
[group1]
IP_OR_HOSTNAME PARAM=VAL PARAM=VAL
IP_OR_HOSTNAME2 

[group2]
IP_OR_HOSTNAME3 PARAM=VAL PARAM=VAL
IP_OR_HOSTNAME4 

[group3:children]
group1
group2
\end{lstlisting}
\newpage

%-------------------------------------------------------------------------------
\end{document}
