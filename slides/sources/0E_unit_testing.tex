% python classes slides - classes
% (c) 2014 Kostiantyn Danylov aka koder
% koder.mail@gmail.com
% distributed under CC-BY licence
% http://creativecommons.org/licenses/by/3.0/deed.en

\documentclass{article}
% XeLaTeX
\usepackage{xltxtra}
\usepackage{xunicode}
\usepackage{listings}
\usepackage[landscape]{geometry}

% Fonts
\setmainfont{DejaVu Sans} %{Arial}
\newfontfamily\cyrillicfont{Nimbus Roman No9 L} %{Arial}
\setmonofont{Courier New}
%\setmonofont{Ubuntu Mono}

%\setmonofont{DejaVu Sans Mono}

% Lang
\usepackage{polyglossia}
\setmainlanguage{russian}
\setotherlanguage{english}
\usepackage[dvipsnames,table]{xcolor}


\ifx\pdfoutput\undefined
\usepackage{graphicx}
\else
\usepackage[pdftex]{graphicx}
\fi

\lstset{
	language=python,
	keywordstyle=\color{Emerald},%\texttt, 
	commentstyle=\color{OliveGreen},%\texttt,
	stringstyle=\color{Bittersweet},%\texttt,
	tabsize=4,
	numbers=left,
	xleftmargin=10pt,
	morekeywords={with,as},	
	numberstyle=\large,
	%identifierstyle=\texttt,
	%basicstyle=\texttt,
}

\usepackage{hyperref}

\hypersetup{
	colorlinks=true,
	urlcolor=blue
}

\usepackage{float}
%\floatstyle{boxed} 
%\restylefloat{figure}
\usepackage[normalem]{ulem}

\input{files/python_cmds}
\begin{document}
\LARGE

%-------------------------------------------------------------------------------
\begin{center}Юнит-тесты\end{center}
    Код, который проверяет код, что бы уволить тестировщиков.
\begin{verbatim}
    def my_test():
    	if the_answer_to_the_ultimate_question() != 42:
            raise SomeError("It does not look like anything")
\end{verbatim}
\newpage

%-------------------------------------------------------------------------------
\begin{center}Структура\end{center}
\begin{itemize}
    \item Дерево файлов в отдельной папке(tests), которая частично моделирует
          структуру основного проекта(если только не ява)
    \item Версионируется вместе с кодом
\end{itemize}
\newpage

%-------------------------------------------------------------------------------
\begin{center}Тесты\end{center}
\begin{itemize}
    \item Методы в классах или функции
    \item Инициализация
	\item Код теста - выполняется действие и результат сверяется с требуемым
\end{itemize}
\newpage

%-------------------------------------------------------------------------------
\begin{center}Фреймворк\end{center}
\begin{itemize}
    \item Поиск тестов
    \item Инициализация, контроль исполнения
	\item Отчеты, взаимодействие с CI/CD
    \item Библиотека вспомагательных функций
    \item Контроль логов
    \item Вызов отладчика по ошибке, etc
    \item Шедулинг исполнения
    \item Профилирования
    \item История исполнения
    \item ....
\end{itemize}
\newpage

%-------------------------------------------------------------------------------
\begin{center}\lstinline!assert & AssertionError!\end{center}
\begin{itemize}
	\item AssertionError - lingua franca юнит-тестов
    \item Все тестирующие ф-ции выбрасывают их, все фреймворки - ловят
	\item \lstinline!assert expr[, msg] - assert x == 3!
\end{itemize}

\begin{verbatim}
	x = 1
	y = 2
    assert x == y, "x(={}) must be equals y(={})".format(x, y)
    assert the_answer_to_the_ultimate_question() == 42
\end{verbatim}

%-------------------------------------------------------------------------------
\begin{center}Фреймворки\end{center}
\begin{itemize}
	\item unittest - stdlib, тяжелое наследия явы, в 3 починили слегка
    \item nose - http://nose.readthedocs.io/en/latest/, использует unittest
	\item pytest - https://docs.pytest.org/en/latest
\end{itemize}

%-------------------------------------------------------------------------------
\begin{center}Проверка\end{center}
\begin{lstlisting}
    assert response == 250   <<< pytest
    self.assertEqual(response, 250)  <<< unittest
    ok(responce) == 250  <<< oktest
\end{lstlisting}
\newpage

%-------------------------------------------------------------------------------
\begin{center}Инициализация и очистка(nose+unittest)\end{center}
\begin{itemize}
	\item setUpClass
    \item setUp
	\item tearDown
    \item tearDownClass
\end{itemize}
\newpage

%-------------------------------------------------------------------------------
\begin{center}Инициализация pytest\end{center}
\begin{itemize}
	\item setup() - module level
    \item teardown() - module level
	\item setup\_module(module)
    \item teardown\_module(module)
    \item setup\_function(function)
    \item teardown\_function(function)
    \item setup\_class(cls)
    \item teardown\_class(cls)
\end{itemize}
\newpage

%-------------------------------------------------------------------------------
\begin{center}pytest fuxtures\end{center}
\begin{lstlisting}
@pytest.fixture(params=[v1, v2])
def resource_setup(request):
    print("resource_setup")
    yield (some_data, request.param)
    print("resource_teardown")

@pytest.mark.slowtest
def some_test(resource_setup):
    pass
\end{lstlisting}
Совместное использование fuctures
\newpage

%-------------------------------------------------------------------------------
\begin{center}Поиск и исполнение тестов\end{center}
\begin{lstlisting}
$ pytest test_mod.py::test_func
$ pytest -s test_mod.py::TestClass::test_method
$ pytest -v -m webtest
\end{lstlisting}
\newpage

%-------------------------------------------------------------------------------
\begin{center}Описывайте записимости явно\end{center}
\begin{lstlisting}
def some_test():
    some_initialization("x", "y", "z")
    test_code()
    some_deinitialization()

def some_test2():
    with some_initialization("x", "y", "z"):
        test_code()

@require_some_init("x", "y", "z")
def some_test3():
    test_code()
\end{lstlisting}
\newpage


%-------------------------------------------------------------------------------
\begin{center} \href{http://www.kuwata-lab.com/oktest/oktest-py_users-guide.html}{oktest} \end{center}
\begin{lstlisting}
	from oktest import ok

	def simple_test():
		ok(1) == 1
		ok([]).is_a(list)
		ok("/tmp/x.txt").is_file()
\end{lstlisting}
\newpage

%-------------------------------------------------------------------------------
\begin{center} mock \end{center}
\begin{lstlisting}
	import os
	import glob

	def remove_tmp_files():
		for fname in glob.glob("/tmp/*.tmp"):
			os.unlink(fname)
\end{lstlisting}
\newpage

%-------------------------------------------------------------------------------
\begin{center}\end{center}
\begin{lstlisting}
	import mock

	def glob_mock(path):
		return ["/tmp/x.tmp", "/tmp/y.sh"]

	all_unlinked = []
	def unlink_mock(path):
		return all_unlinked.append(path)

	@mock.patch("os.unlink", unlink_mock)
	@mock.patch("glob.glob", glob_mock)
	def test_remove_tmp():
		remove_tmp_files()
		ok(all_unlinked) == ["/tmp/x.tmp"]
\end{lstlisting}
\newpage
\end{document}
