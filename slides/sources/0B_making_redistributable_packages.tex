% python classes slides - redistributable packages
% (c) 2012 Kostiantyn Danylov aka koder 
% koder.mail@gmail.com
% distributed under CC-BY licence
% http://creativecommons.org/licenses/by/3.0/deed.en

\documentclass{article}
% XeLaTeX
\usepackage{xltxtra}
\usepackage{xunicode}
\usepackage{listings}
\usepackage[landscape]{geometry}

% Fonts
\setmainfont{DejaVu Sans} %{Arial}
\newfontfamily\cyrillicfont{Nimbus Roman No9 L} %{Arial}
\setmonofont{Courier New}
%\setmonofont{Ubuntu Mono}

%\setmonofont{DejaVu Sans Mono}

% Lang
\usepackage{polyglossia}
\setmainlanguage{russian}
\setotherlanguage{english}
\usepackage[dvipsnames,table]{xcolor}


\ifx\pdfoutput\undefined
\usepackage{graphicx}
\else
\usepackage[pdftex]{graphicx}
\fi

\lstset{
	language=python,
	keywordstyle=\color{Emerald},%\texttt, 
	commentstyle=\color{OliveGreen},%\texttt,
	stringstyle=\color{Bittersweet},%\texttt,
	tabsize=4,
	numbers=left,
	xleftmargin=10pt,
	morekeywords={with,as},	
	numberstyle=\large,
	%identifierstyle=\texttt,
	%basicstyle=\texttt,
}

\usepackage{hyperref}

\hypersetup{
	colorlinks=true,
	urlcolor=blue
}

\usepackage{float}
%\floatstyle{boxed} 
%\restylefloat{figure}
\usepackage[normalem]{ulem}

\input{files/python_cmds}

\begin{document}
\LARGE

%-------------------------------------------------------------------------------
\begin{center} Chronology Of Packaging \end{center}
    slides from \href{http://ziade.org/2012/11/17/chronology-of-packaging/}
                        {Tarek Ziade - Chronology Of Packaging}
    \includegraphics[scale=1.0]{images/packaging-history-part1.png}\\
    \includegraphics[scale=1.0]{images/packaging-history-part2.png}
\newpage

%-------------------------------------------------------------------------------
\begin{center} pypi \end{center}
crate.io
\newpage

%-------------------------------------------------------------------------------
\begin{center} \href{http://pypi.python.org/pypi/modern-package-template/1.0}
                    {modern-package-template} \end{center}

\newpage

%-------------------------------------------------------------------------------
\begin{center} distutils \end{center}
\newpage

%-------------------------------------------------------------------------------
\begin{center} distribute \end{center}
\newpage 

%-------------------------------------------------------------------------------
\begin{center} easy\_install/pip \end{center}
\newpage 

%-------------------------------------------------------------------------------
\begin{center} C Python Extension \end{center}
\newpage

%-------------------------------------------------------------------------------
\begin{center} PyCXX \end{center}
\begin{lstlisting}
    d = {}
    d["a"] = 1
    d["b"] = 2
    alist = d.keys()
    print alist
\end{lstlisting}
\lstset{language=C++}
\begin{lstlisting}
    Py::Dict d;
    Py::List alist;
    d["a"] = Py::Long(1);
    d["b"] = Py::Long(2);
    alist = d.keys();
    std::cout << alist << std::endl;
\end{lstlisting}
\lstset{language=python}
\newpage

%-------------------------------------------------------------------------------
\begin{center} boost.python \end{center}
\lstset{language=C++}
\begin{lstlisting}
    #include <boost/python.hpp>

    char const* greet()
    {
       return "hello, world";
    }

    BOOST_PYTHON_MODULE(hello_ext)
    {
        using namespace boost::python;
        def("greet", greet);
    }
\end{lstlisting}
\lstset{language=python}
\newpage

%-------------------------------------------------------------------------------
\begin{center} SWIG/SIP/...... \end{center}
\newpage

%-------------------------------------------------------------------------------
\begin{center} ctypes \end{center}
\newpage

%-------------------------------------------------------------------------------
\begin{center} cffi \end{center}
\newpage

%-------------------------------------------------------------------------------
\begin{center} \href{http://www.cython.org/}{Cython} \end{center}
\begin{itemize}
    \item Форк \href{http://www.cosc.canterbury.ac.nz/greg.ewing/python/Pyrex/}{Pyrex}
    \item Позволяет смешивать python и C код в одном файле/функции/классе
    \item Компилирует полученный модуль в расширение для python
    \item Чем больше информации о типах - тем выше скорость исполнения
\end{itemize}
\newpage

%-------------------------------------------------------------------------------
%-------------------------------------------------------------------------------
\end{document}
