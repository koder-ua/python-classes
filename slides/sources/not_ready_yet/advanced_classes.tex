%  python classes clides - advanced OOP
% (c) 2012 Kostiantyn Danylov aka koder 
% koder.mail@gmail.com
% distributed under CC-BY licence
% http://creativecommons.org/licenses/by/3.0/deed.en

\documentclass{article}
% XeLaTeX
\usepackage{xltxtra}
\usepackage{xunicode}
\usepackage{listings}
\usepackage[landscape]{geometry}

% Fonts
\setmainfont{DejaVu Sans} %{Arial}
\newfontfamily\cyrillicfont{Nimbus Roman No9 L} %{Arial}
\setmonofont{Courier New}
%\setmonofont{Ubuntu Mono}

%\setmonofont{DejaVu Sans Mono}

% Lang
\usepackage{polyglossia}
\setmainlanguage{russian}
\setotherlanguage{english}
\usepackage[dvipsnames,table]{xcolor}


\ifx\pdfoutput\undefined
\usepackage{graphicx}
\else
\usepackage[pdftex]{graphicx}
\fi

\lstset{
	language=python,
	keywordstyle=\color{Emerald},%\texttt, 
	commentstyle=\color{OliveGreen},%\texttt,
	stringstyle=\color{Bittersweet},%\texttt,
	tabsize=4,
	numbers=left,
	xleftmargin=10pt,
	morekeywords={with,as},	
	numberstyle=\large,
	%identifierstyle=\texttt,
	%basicstyle=\texttt,
}

\usepackage{hyperref}

\hypersetup{
	colorlinks=true,
	urlcolor=blue
}

\usepackage{float}
%\floatstyle{boxed} 
%\restylefloat{figure}
\usepackage[normalem]{ulem}

\input{files/python_cmds}
\begin{document}
\LARGE

%---------------------------------------------------------------------------------
\center{TODO}
\begin{itemize}
    \item \_\_slots\_\_
    \item type conversion
    \item Метаклассы
    \item property
    \item Поиск атрибута
    \item weakref + \_\_weakref\_\_ + имитация \_\_del\_\_
    \item implementing protocols - container, contextmanager, generator
    \item приведениие типов, как int(x)
    \item ABC
    \item zope.interfaces
    \item PEAK PyProtocols
\end{itemize}
\newpage

%-------------------------------------------------------------------------------
\center{Внутреннее устройство экземпляра}
\begin{itemize}
\item Экземпляр состоит из \lstinline!__dict__! - словаря атрибутов и 
        \lstinline!__class__! - ссылки на свой класс (тип)
\item \lstinline!a.b == a.__dict__['b'] _or_ a.__class__.__dict__['b'] _or_ ..!
\end{itemize}
\newpage

%-------------------------------------------------------------------------------
\center{Внутреннее устройство класса}
\begin{itemize}
\item \lstinline!class! это синтаксический сахар
\end{itemize}
{
\LARGE \vspace{15pt}
\begin{lstlisting}
    def method(self, val):
        "method documentation"
        self.some_field = val
        return val ** 0.3

    Simple = type('Simple', 
                  (ParentClass,), 
                  {'method':method, 
                   '__doc__': "class documentation"}
                 )
\end{lstlisting}
}
\newpage

%-------------------------------------------------------------------------------
\center{Внутреннее устройство класса}
\begin{itemize}

\item Блок внутри \lstinline!class! исполняется и 
        полученный \lstinline!locals()! становится 
            \lstinline!__dict__! класса
\item Класс может быть модифицирован в любое время и эта модификация 
        тут-же отобразится на всех экземплярах 

\end{itemize}
\newpage

%---------------------------------------------------------------------------------
\center{Множественное Наследование}
{
\Large \vspace{15pt}
\begin{lstlisting}
class A(object):
    def draw(self, pt):
        some_action()

    def clear(self):
        pass

class B(A):
    def draw(self, pt):
        A.draw(self, pt)

class C(A):
    def draw(self, pt):
        A.draw(self, pt)

    def clear(self):
        pass

class D(B, C):
    def draw(self, pt):
        B.draw(self, pt)
        C.draw(self, pt)

print A.__mro__ 
# (<class '__main__.A'>, <type 'object'>)

print B.__mro__ 
# (<class '__main__.B'>, <class '__main__.A'>, <type 'object'>)

print C.__mro__ 
# (<class '__main__.C'>, <class '__main__.A'>, <type 'object'>)

\end{lstlisting}
%print D.__mro__ 
% (<class '__main__.D'>, <class '__main__.B'>, 
%  <class '__main__.C'>, <class '__main__.A'>, 
% <type 'object'>)
}
\newpage
%-------------------------------------------------------------------------------
\center{super}
\begin{lstlisting}
class A(object):
    def draw(self, pt):
        some_action()

class B(A):
    def draw(self, pt):
        super(B, self).draw(pt)

class C(A):
    def draw(self, pt):
        super(C, self).draw(pt)

class D(B, C):
    def draw(self, pt):
        super(D, self).draw(pt)
\end{lstlisting}
\newpage

%-------------------------------------------------------------------------------

\begin{lstlisting}
class B(object):
    def draw(self, pt):
        pass

class C(object):
    def draw(self, pt):
        pass

class D(B, C):
    def draw(self, pt):
        super(D, self).draw(pt) # ?????
\end{lstlisting}
\newpage

%-------------------------------------------------------------------------------
\end{document}
