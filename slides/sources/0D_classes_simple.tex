% python classes slides - classes
% (c) 2014 Kostiantyn Danylov aka koder 
% koder.mail@gmail.com
% distributed under CC-BY licence
% http://creativecommons.org/licenses/by/3.0/deed.en

\documentclass{article}
% XeLaTeX
\usepackage{xltxtra}
\usepackage{xunicode}
\usepackage{listings}
\usepackage[landscape]{geometry}

% Fonts
\setmainfont{DejaVu Sans} %{Arial}
\newfontfamily\cyrillicfont{Nimbus Roman No9 L} %{Arial}
\setmonofont{Courier New}
%\setmonofont{Ubuntu Mono}

%\setmonofont{DejaVu Sans Mono}

% Lang
\usepackage{polyglossia}
\setmainlanguage{russian}
\setotherlanguage{english}
\usepackage[dvipsnames,table]{xcolor}


\ifx\pdfoutput\undefined
\usepackage{graphicx}
\else
\usepackage[pdftex]{graphicx}
\fi

\lstset{
	language=python,
	keywordstyle=\color{Emerald},%\texttt, 
	commentstyle=\color{OliveGreen},%\texttt,
	stringstyle=\color{Bittersweet},%\texttt,
	tabsize=4,
	numbers=left,
	xleftmargin=10pt,
	morekeywords={with,as},	
	numberstyle=\large,
	%identifierstyle=\texttt,
	%basicstyle=\texttt,
}

\usepackage{hyperref}

\hypersetup{
	colorlinks=true,
	urlcolor=blue
}

\usepackage{float}
%\floatstyle{boxed} 
%\restylefloat{figure}
\usepackage[normalem]{ulem}

\input{files/python_cmds}
\begin{document}
\LARGE

%-------------------------------------------------------------------------------
\begin{center} Термины \end{center}
\begin{itemize}
    \item Интерфейс - набор всего, что видно наружу из X. Автомобиль - интерфейс.
    \item Интерфейсы могут наследовать и расширять друг друга. Ручная коробка передач и автоматическая.
    \item Классы - конкретные типа объектов, реализующие интерфейс(ы).
    \item Ауди, БТР, жигули. 
    \item Классы могут наследовать и расширять друг друга.
    \item Класс только определяет тип и реализацию методы интерфейса.
    \item Объекты - экземпляры класса. Объекты содержат конкретные данные.
    \item Все объекты конкретного класса используют общую реализацию функциональности, 
           но передают в нее свои конкретные данные (текущая скорость, коефициенты трения, 
           количество и тип топлива).
\end{itemize}
\newpage

%-------------------------------------------------------------------------------
\begin{center} Что куда относится \end{center}
\begin{itemize}
    \item Скорость автомобиля
    \item Управление рулем/рычагами
    \item Тип коробки-автомата (робот/вариатор/...)
    \item Пробег двигателя
    \item Максимальное ускорение
    \item Тип привода
    \item Угол поворота рулевого колеса
    \item Уровень топлива
\end{itemize}
\newpage


%-------------------------------------------------------------------------------
\begin{center} Код для автомобиля \end{center}
\begin{lstlisting}
    auto = ((x, y), (vec_x, vec_y), acceleration_level)

    def set_accel_level1(auto, al):
        pass

    def set_deccel_level1(auto, dl):
        pass

    def set_wheel_pos1(auto, wp):
        pass
\end{lstlisting}
\newpage

%-------------------------------------------------------------------------------
\begin{center} Код для автомобиля \end{center}
\begin{equation}
    x = x + vec_x
\end{equation}    
\begin{equation}
    y = y + vec_y
\end{equation}    
\begin{equation}
    speed = \sqrt{vec_x^2 + vec_y^2}
\end{equation}    
\begin{equation}
    vec_x = vec_x + max\_accel * accel\_lev + air\_res * speed ^{3} / mass
\end{equation}    

\begin{lstlisting}
    def tick1(auto):
        #...
\end{lstlisting}
\newpage

%-------------------------------------------------------------------------------
\begin{center} Функция, обрабатывающая автомобиль \end{center}
\begin{lstlisting}
    auto = ((x, y), (vec_x, vec_y), acceleration_level, tp)
    def make_trip(auto, path):
        #.....
        if tp == "audi":
            auto = set_accel_level1(auto, ..)
        else tp == "zaz":
            auto = set_accel_level2(auto, ..)
            #....
\end{lstlisting}
\newpage

%-------------------------------------------------------------------------------
\begin{center} Функция, обрабатывающая автомобиль \end{center}
\begin{lstlisting}
    auto = ((x, y), (vec_x, vec_y), acceleration_level, tp, accel_func, ...)
    def make_trip(auto, path):
        #...
        accel_func = auto[4]
        #...
        auto = accel_func(auto, ..)


\end{lstlisting}
\newpage

%-------------------------------------------------------------------------------
\begin{center} Функция, создающая автомобиль \end{center}
\begin{lstlisting}
    def make_zaz(x, y, vec_x=0, vec_y=0, acceleration_level=0):
        return ((x, y), (vec_x, vec_y), acceleration_level, tp, accel_func, ...)

    auto = make_zaz(1, 2)
\end{lstlisting}
\newpage

%-------------------------------------------------------------------------------
\begin{center} Классы \end{center}
\begin{lstlisting}
    class Auto(object):
        def set_accel_level(self, al):
            pass

        def set_deccel_level(self, dl):
            pass

        def set_wheel_pos(self, wp):
            pass

        def speed(self):
            pass

\end{lstlisting}
\newpage

%-------------------------------------------------------------------------------
\begin{center} Классы \end{center}
\begin{lstlisting}
    class ZAZ(Auto):
        def __init__(self, x, y):
            self.

        def set_accel_level(self, al):
            pass

        def set_deccel_level(self, dl):
            pass

        def set_wheel_pos(self, wp):
            pass

        def tick(self):
            pass
\end{lstlisting}
\newpage

%-------------------------------------------------------------------------------
\begin{center} Классы \end{center}
\begin{lstlisting}
    auto = ZAZ(100, 100)
    auto.set_accel_level(1)
    auto.tick()
    auto.tick()
    print "Speed after two ticks is", auto.speed()
\end{lstlisting}
\newpage

%-------------------------------------------------------------------------------
\begin{center} Классы \end{center}
\begin{itemize}
    \item разогнать автомобиль до 60 км/час
    \item проехать 6км
    \item затормозить до 0
\end{itemize}

\begin{lstlisting}
def make_trip(auto, path):
    pass
\end{lstlisting}
\newpage

%-------------------------------------------------------------------------------
\end{document}
