% python classes slides - basic data types
% (c) 2012 Kostiantyn Danylov aka koder 
% koder.mail@gmail.com
% distributed under CC-BY licence
% http://creativecommons.org/licenses/by/3.0/deed.en

\documentclass{article}
% XeLaTeX
\usepackage{xltxtra}
\usepackage{xunicode}
\usepackage{listings}
\usepackage[landscape]{geometry}

% Fonts
\setmainfont{DejaVu Sans} %{Arial}
\newfontfamily\cyrillicfont{Nimbus Roman No9 L} %{Arial}
\setmonofont{Courier New}
%\setmonofont{Ubuntu Mono}

%\setmonofont{DejaVu Sans Mono}

% Lang
\usepackage{polyglossia}
\setmainlanguage{russian}
\setotherlanguage{english}
\usepackage[dvipsnames,table]{xcolor}


\ifx\pdfoutput\undefined
\usepackage{graphicx}
\else
\usepackage[pdftex]{graphicx}
\fi

\lstset{
	language=python,
	keywordstyle=\color{Emerald},%\texttt, 
	commentstyle=\color{OliveGreen},%\texttt,
	stringstyle=\color{Bittersweet},%\texttt,
	tabsize=4,
	numbers=left,
	xleftmargin=10pt,
	morekeywords={with,as},	
	numberstyle=\large,
	%identifierstyle=\texttt,
	%basicstyle=\texttt,
}

\usepackage{hyperref}

\hypersetup{
	colorlinks=true,
	urlcolor=blue
}

\usepackage{float}
%\floatstyle{boxed} 
%\restylefloat{figure}
\usepackage[normalem]{ulem}

\input{files/python_cmds}
\begin{document}
\LARGE
%-------------------------------------------------------------------------------
\begin{lstlisting}
stack.insert(0, val)
stack.pop(0)
\end{lstlisting}
\newpage

%-------------------------------------------------------------------------------
\begin{lstlisting}
def load_file(fname):
    lines = []
    with open(fname) as fd:
        for line in fd:
            #....
            lines.append(processed_line)
    return lines
\end{lstlisting}
\newpage

%-------------------------------------------------------------------------------
\begin{lstlisting}
def load_file2(fname):
    with open(fname) as fd:
        for line in fd:
            #....
            yield processed_line

\end{lstlisting}
\newpage

%-------------------------------------------------------------------------------
\begin{lstlisting}
def load_file3(fd):
    for line in fd:
        #....
        yield processed_line

def execute(fname):
    for cmd in load_file3(fname):
        #.....
\end{lstlisting}
\newpage

%-------------------------------------------------------------------------------
\begin{lstlisting}
class Forth(object):
    stack = []


class Forth(object):
    def __init__(self):
        self._stack = []


class Forth(object):
    def __init__(self, statements):
        self.current_parameter = None

    def execute(self):
        for ....:
            #....
            self.current_parameter = some_data
            run_command
\end{lstlisting}
\newpage

%-------------------------------------------------------------------------------
\begin{lstlisting}
a, b = self.stack.pop(), self.stack.pop()
a[func1()] = func2()


cmd = s.split()[0]
param = s.split()[1]
\end{lstlisting}
\newpage

%-------------------------------------------------------------------------------
\begin{lstlisting}

class Forth(object):
    # ....
    def run(self):
        for s in self._statements:
            if hasattr(self, s.split()[0]):
                method = getattr(self, s.split()[0])
                method()
\end{lstlisting}
\newpage

%-------------------------------------------------------------------------------
\begin{center}LEAP vs EAFP\end{center}
\begin{lstlisting}
def add_LEAP(stack):
    if len(stack) < 2:
        raise SomeNiceClass()
    stack.append(stack.pop() + stack.pop())

def add_EAFP(stack):
    stack.append(stack.pop() + stack.pop())
\end{lstlisting}
\newpage

%-------------------------------------------------------------------------------
\begin{center}Code duplication\end{center}
\begin{lstlisting}
def add():
    p1 = convert_to_number(pop())
    p2 = convert_to_number(pop())
    ...

def sub():
    p1 = convert_to_number(pop())
    p2 = convert_to_number(pop())
    ....
\end{lstlisting}
\newpage

%-------------------------------------------------------------------------------
\begin{center}SOLID (SRP, OCR, LSP, ISP, DIP), YAGNI\end{center}
\begin{lstlisting}
class Forth(object):
    def __init__(self, statements): 

    def put(self):

    def pop(self):

    def add(self):

    def sub(self):

    def print_(self):

    def run(self):

def execute(file_name):
    Forth(source_code).run()
\end{lstlisting}
\newpage

%-------------------------------------------------------------------------------
\begin{lstlisting}
class Forth(__Stack__):
    def put(self):
        pass

    def pop(self):
        pass
\end{lstlisting}
\newpage

%-------------------------------------------------------------------------------
\begin{center}Lexer => Parser => Compiler => Executor\end{center}
\begin{center}Lexer + Parser\end{center}
\begin{lstlisting}
    def parse_file(fd):
        for lineno, raw_line in enumerate(fd):
            line = raw_line.strip()
            if line == "" or line.startswith("#"):
                continue
            try:
                if " " not in line:
                    yield lineno, (line, None)
                else:
                    cmd, param = line.split(" ", 1)
                    if param.startswith('"') and param.endswith('"'):
                        param = param[1:-1]
                    else:
                        try:
                            param = int(param)
                        except ValueError:
                            param = float(param)
                    yield lineno, (cmd, param)
            except Exception as x:
                print >>sys.stderr, "Parse error at line", lineno
                raise
\end{lstlisting}
\newpage

%-------------------------------------------------------------------------------
\begin{center}Compiler + Executor\end{center}
\begin{lstlisting}
def execute(stack, fd):
    for lineno, (cmd, param) in parse_file(fd):
        if cmd == "put":
            stack.append(param)
        elif cmd == "add":
            stack.append(stack.pop() + stack.pop())
        elif cmd == "sub":
            stack.append(stack.pop() - stack.pop())
        elif cmd == "print":
            print stack.pop()
        else:
            raise ValueError("Unknown command {} at line {}".format(cmd, lineno))
\end{lstlisting}
\newpage

%-------------------------------------------------------------------------------
\begin{center}Command\end{center}
\begin{lstlisting}
class Command(object):
    name = None
    param_count = None
    minimum_stack_size = None
    def __init__(self, **params):
        self.params = params
    def execute(self, stack):
        if len(stack) < self.minimum_stack_size:

class Add(Command):
    name = 'add'
    param_count = 0
    minimum_stack_size = 2
    def execute(self, stack):
        stack.append(stack.pop() + stack.pop())

all_commands = {Add.name: Add, ...}
\end{lstlisting}
\newpage

%-------------------------------------------------------------------------------
\begin{center}Command\end{center}
\begin{lstlisting}
class Command_0_0(object):
    def __init__(self, **params):
        self.name = params

    def execute(self, stack):
        if len(stack) < self.minimum_stack_size:
            .....

class Add_0_2(Command):
    pass

all_command_classes = [Add_0_2, ..]
all_commands = {}

\end{lstlisting}
\newpage

%-------------------------------------------------------------------------------
\begin{center}Command\end{center}
\begin{lstlisting}
for cmd in all_command_classes:
    uname, stack_sz, num_params = cmd.__name__.split("_")
    cmd.name = uname.lower()
    cmd.param_count = int(num_params)
    ...
    all_commands[cmd.name] = cmd
\end{lstlisting}
\newpage

%-------------------------------------------------------------------------------
\begin{center}Command\end{center}
\begin{lstlisting}
def add_2_0(stack):
    #....

def add_2_0(stack, s1, s2):
    #...

def put(stack, p1):
    stack.append(p1)
\end{lstlisting}
\newpage

%-------------------------------------------------------------------------------
\end{document}
