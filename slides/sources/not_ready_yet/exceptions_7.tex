% python classes slides - exceptions
% (c) 2012 Kostiantyn Danylov aka koder 
% koder.mail@gmail.com
% distributed under CC-BY licence
% http://creativecommons.org/licenses/by/3.0/deed.en

\documentclass{article}
% XeLaTeX
\usepackage{xltxtra}
\usepackage{xunicode}
\usepackage{listings}
\usepackage[landscape]{geometry}

% Fonts
\setmainfont{DejaVu Sans} %{Arial}
\newfontfamily\cyrillicfont{Nimbus Roman No9 L} %{Arial}
\setmonofont{Courier New}
%\setmonofont{Ubuntu Mono}

%\setmonofont{DejaVu Sans Mono}

% Lang
\usepackage{polyglossia}
\setmainlanguage{russian}
\setotherlanguage{english}
\usepackage[dvipsnames,table]{xcolor}


\ifx\pdfoutput\undefined
\usepackage{graphicx}
\else
\usepackage[pdftex]{graphicx}
\fi

\lstset{
	language=python,
	keywordstyle=\color{Emerald},%\texttt, 
	commentstyle=\color{OliveGreen},%\texttt,
	stringstyle=\color{Bittersweet},%\texttt,
	tabsize=4,
	numbers=left,
	xleftmargin=10pt,
	morekeywords={with,as},	
	numberstyle=\large,
	%identifierstyle=\texttt,
	%basicstyle=\texttt,
}

\usepackage{hyperref}

\hypersetup{
	colorlinks=true,
	urlcolor=blue
}

\usepackage{float}
%\floatstyle{boxed} 
%\restylefloat{figure}
\usepackage[normalem]{ulem}

\input{files/python_cmds}

\begin{document}
\LARGE

%-------------------------------------------------------------------------------
\begin{center} Пример - обработка ошибок \end{center}
Как написать надежную обработку ошибок?
\begin{itemize}
	\item Для каждой функции выделить специальное значение для индикации ошибки
	\item Проверять результат каждой функции и в случае ошибки обрабатывать
	       ее или передавать дальше, если обработка в текущей точке невозможна
\end{itemize}

\begin{lstlisting}
	def do_some_work(name, vals):
		if not isinstance(name, basestring):
			return None
		#......
		return res

	def f2():
		res = do_some_work("1231", [1, 2])
		if res is None:
			return None
		.....
\end{lstlisting}
\newpage

%-------------------------------------------------------------------------------
\begin{center} Проблема 1 \end{center}
\begin{itemize}
	\item Нужно помнить какое значение возвращает эта функция при ошибке.
	\item Нужно хранить дополнительно информацию о ошибке
	\item
	\item Возвращать из каждой функции пару
\end{itemize}
\newpage

%-------------------------------------------------------------------------------
\begin{center} Решение \end{center}
\begin{itemize}
	\item Возвращать из каждой функции тройку (is_ok, tb, result)
	\item Превратить return res в return (True, [], res)
	\item Превратить return err в return (False, [fname], err_info)
	\item Проверять на выходе из каждой функции результат
	\item Что-бы информация имела смысл - дописывать текущую точку
\end{itemize}
\newpage

%-------------------------------------------------------------------------------
\begin{lstlisting}
	def do_some_work(name, vals):
		if not isinstance(name, basestring):
			return (False, ["do_some_work"], "name should be a string")
		#......
		return (True, None, res)

	def f2():
		is_ok, res, stack = do_some_work("1231", [1, 2])
		if not is_ok:
			return (False, stack + ["f2"], res)
		.....
\end{lstlisting}
\newpage

%-------------------------------------------------------------------------------
\begin{center} Еще проблемы \end{center}
\begin{itemize}
	\item Это очень сильно загрязняет код
	\item Невозможны вложенные вызовы функций
	\item Такие ошибки чаще не проверяют (printf)
	\item Если не проверить ошибку, то проблема 
			возникнет в непредсказуемом месте кода
	\item В обработчиках ошибок нужно фильтровать ошибки по типу
	\item
	\item Вспомагательный код очень простой - его 
		  генерацию можно переложить на компилятор
\end{itemize}
\newpage

%-------------------------------------------------------------------------------
\begin{center} Именно это и делают современные языки \end{center}
\begin{lstlisting}
	#return (False, val, ["fname"])
	raise val

	#if not is_ok and this_is_my_exception(val):
	try:
		code
	except ExceptionType as x:
		error_process_code
\end{lstlisting}
\newpage

%-------------------------------------------------------------------------------
\begin{center} Исключения \end{center}
\begin{itemize}
	\item Исключение – это событие, после которого дальнейшее продолжение 
			работы в данной точке бессмысленно. По итогу такого события 
			генерируется объект-исключение, и исполнение передается обработчику 
			ошибок этого типа
	\item Пример – деление на 0, выбрасывается ошибка ZeroDivisionError
	\item Исключения помогают упростить код, убрав из него множество 
			проверок и значительно облегчить восстановление программы после сбоя
	\item Исключения упрощают доставку информации о ошибке от той точке, 
			в которой она возникла к той точке где она может быть обработанны
\end{itemize}
\newpage

%-------------------------------------------------------------------------------

\begin{center} Исключения \end{center}
\begin{lstlisting}
	try:
		block1
	except tp2 as var2:
		block2
	except (tp3, tp4) as var3:
		block3
	else:
		block5
	finally:
		block4

\end{lstlisting}
\newpage

%-------------------------------------------------------------------------------
\begin{center} Исключения \end{center}
\begin{lstlisting}
	try:
		raise tp2("xxx")  # <<<<
	except tp2 as var2:
		block2            # <<<<
	except (tp3, tp4) as var3:
		block3
	else:
		block5
	finally:
		block4            # <<<<
\end{lstlisting}
\newpage

%-------------------------------------------------------------------------------
\begin{center} Исключения \end{center}
\begin{lstlisting}
	try:
		pass           # <<<
	except tp2 as var2:
		block2
	except (tp3, tp4) as var3:
		block3
	else:
		block5         # <<<
	finally:
		block4         # <<<
\end{lstlisting}
\newpage

%-------------------------------------------------------------------------------
\begin{center} Исключения \end{center}
\begin{lstlisting}
	def f1(t, d, x, y):
		if t – d  == 0:
		    return None
		else:
		    t1 = ((x + y) / (t - d))
		    if t1 == 0:
		        return None
		    else:
		        return 1 / ((x + y) / (t - d))

	def f2(t, d, x, y):
		try:
		    return 1 / ((x + y) / (t - d))
		except ZeroDivisionError:
		    return None
\end{lstlisting}
\newpage

%-------------------------------------------------------------------------------
\begin{center} Исключения. raise \end{center}
\begin{itemize}
	\item \lstinline!raise ExceptionType("Some message")! порождает исключение
	\item \lstinline!ExceptionType! должно наследовать \lstinline!Exception!
	\item \lstinline!raise! без параметров разрешено только в блоке except. 
		Оно повторно выбрасывает исключение, которое сейчас обрабатывается
\end{itemize}
\begin{lstlisting}
	try:
		func()
	except Exception:
		print "func cause exception"
		raise
\end{lstlisting}
\newpage

%-------------------------------------------------------------------------------
\begin{center} Исключения. traceback \end{center}
	В обработчике исключения \lstinline!sys.exc_info()! возвращает тройку
		(Тип исключения, Объект исключения, Состояние Стека)

\begin{lstlisting}
	try:
	    raise ValueError("ddd")
	except Exception as x:
		tb = sys.exc_info()[2]

	print tb.tb_frame # <frame at 0x....>
	print tb.tb_frame.f_lineno # 4
	print tb.tb_frame.f_code.co_name # '<module>'
	print tb.tb_frame.f_code.co_filename 
		# '<ipython-input-7-492d537cf800>'
	print tb.tb_next # <frame at 0x....> or None
	del tb
\end{lstlisting}
\newpage

%-------------------------------------------------------------------------------
\end{document}

