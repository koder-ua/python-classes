% python classes slides - making classes hierarchy
% (c) 2012 Kostiantyn Danylov aka koder 
% koder.mail@gmail.com
% distributed under CC-BY licence
% http://creativecommons.org/licenses/by/3.0/deed.en

\documentclass{article}
% XeLaTeX
\usepackage{xltxtra}
\usepackage{xunicode}
\usepackage{listings}
\usepackage[landscape]{geometry}

% Fonts
\setmainfont{DejaVu Sans} %{Arial}
\newfontfamily\cyrillicfont{Nimbus Roman No9 L} %{Arial}
\setmonofont{Courier New}
%\setmonofont{Ubuntu Mono}

%\setmonofont{DejaVu Sans Mono}

% Lang
\usepackage{polyglossia}
\setmainlanguage{russian}
\setotherlanguage{english}
\usepackage[dvipsnames,table]{xcolor}


\ifx\pdfoutput\undefined
\usepackage{graphicx}
\else
\usepackage[pdftex]{graphicx}
\fi

\lstset{
	language=python,
	keywordstyle=\color{Emerald},%\texttt, 
	commentstyle=\color{OliveGreen},%\texttt,
	stringstyle=\color{Bittersweet},%\texttt,
	tabsize=4,
	numbers=left,
	xleftmargin=10pt,
	morekeywords={with,as},	
	numberstyle=\large,
	%identifierstyle=\texttt,
	%basicstyle=\texttt,
}

\usepackage{hyperref}

\hypersetup{
	colorlinks=true,
	urlcolor=blue
}

\usepackage{float}
%\floatstyle{boxed} 
%\restylefloat{figure}
\usepackage[normalem]{ulem}

\input{files/python_cmds}
\begin{document}
\LARGE

%-------------------------------------------------------------------------------
\begin{center} ООП проектирование  \end{center}
\begin{itemize}
    \item функции vs методы - в питоне методы единственный способ перегрузить функции
    \item Лисков
    \item has a, is a
    \item Правильные иерархии
    \item Приведение к базовому типу
\end{itemize}
\newpage

%-------------------------------------------------------------------------------
\begin{center} SOLID \end{center}
\begin{itemize}
	\item 	\href{http://ru.wikipedia.org/wiki/SOLID_(%D0%BE%D0%B1%D1%8A%D0%B5%D0%BA%D1%82%D0%BD%D0%BE-%D0%BE%D1%80%D0%B8%D0%B5%D0%BD%D1%82%D0%B8%D1%80%D0%BE%D0%B2%D0%B0%D0%BD%D0%BD%D0%BE%D0%B5_%D0%BF%D1%80%D0%BE%D0%B3%D1%80%D0%B0%D0%BC%D0%BC%D0%B8%D1%80%D0%BE%D0%B2%D0%B0%D0%BD%D0%B8%D0%B5)}{SOLID раз}
    	\href{http://blog.byndyu.ru/2009/10/solid.html}{SOLID два} - только принцип Лисков тут не читать, автор совсем запутался.
    	\href{http://igor.quatrocode.com/2008/09/solid-top-5.html}{SOLID три}
    \item (S) На каждый объект должна быть возложена одна единственная обязанность.
    \item (O) Программные сущности должны быть открыты для расширения, но закрыты для изменения.
    \item (L) Объекты в программе могут быть заменены их наследниками без изменения свойств программы. 
    \item (I) Много специализированных интерфейсов лучше, чем один универсальный.
    \item (D) Зависимости внутри системы строятся на основе абстракций. 
    		  Модули верхнего уровня не зависят от модулей нижнего уровня. Абстракции не должны
    		  зависеть от деталей. Детали должны зависеть от абстракци
\end{itemize}
\newpage

%-------------------------------------------------------------------------------
\begin{center} Принцип подстановки Барбары Лисков \end{center}
    Функции, которые используют ссылки на базовые классы, должны 
    иметь возможность использовать объекты производных классов, не зная об этом.
\begin{center} Следствия \end{center}

\begin{itemize}
    \item Типы полей должны попарно соответствовать принципу Лисков (быть того же типа, 
            или наследника)
    \item Предусловия не могут быть усилены в подклассе - методы B должны принимать 
          те же параметры, что и методы A, и тех же типов или их родителей
    \item Постусловия не могут быть ослаблены в подклассе - методы B должны возвращать
          результаты тех же типов или типов-наследников
\end{itemize}
    Например - при добавлении дополнительного параметра в функцию нужно 
    давать ему значение по умолчанию
\newpage

%-------------------------------------------------------------------------------
\end{document}
