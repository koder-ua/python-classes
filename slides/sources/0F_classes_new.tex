% python classes slides - classes_introduction
% (c) 2014 Kostiantyn Danylov aka koder 
% koder.mail@gmail.com
% distributed under CC-BY licence
% http://creativecommons.org/licenses/by/3.0/deed.en

\documentclass{article}
% XeLaTeX
\usepackage{xltxtra}
\usepackage{xunicode}
\usepackage{listings}
\usepackage[landscape]{geometry}

% Fonts
\setmainfont{DejaVu Sans} %{Arial}
\newfontfamily\cyrillicfont{Nimbus Roman No9 L} %{Arial}
\setmonofont{Courier New}
%\setmonofont{Ubuntu Mono}

%\setmonofont{DejaVu Sans Mono}

% Lang
\usepackage{polyglossia}
\setmainlanguage{russian}
\setotherlanguage{english}
\usepackage[dvipsnames,table]{xcolor}


\ifx\pdfoutput\undefined
\usepackage{graphicx}
\else
\usepackage[pdftex]{graphicx}
\fi

\lstset{
	language=python,
	keywordstyle=\color{Emerald},%\texttt, 
	commentstyle=\color{OliveGreen},%\texttt,
	stringstyle=\color{Bittersweet},%\texttt,
	tabsize=4,
	numbers=left,
	xleftmargin=10pt,
	morekeywords={with,as},	
	numberstyle=\large,
	%identifierstyle=\texttt,
	%basicstyle=\texttt,
}

\usepackage{hyperref}

\hypersetup{
	colorlinks=true,
	urlcolor=blue
}

\usepackage{float}
%\floatstyle{boxed} 
%\restylefloat{figure}
\usepackage[normalem]{ulem}

\input{files/python_cmds}

\begin{document}
\LARGE
%-------------------------------------------------------------------------------
\begin{center} ООП \end{center}
    \begin{itemize}
         \item Повторное использование кода
         \item Инкапсуляция
         \item Наследование
         \item Полиморфизм
     \end{itemize} 
\newpage
%-------------------------------------------------------------------------------
\begin{center} \includegraphics[scale=5]{images/Hamatsa_shaman2.jpg} \end{center}
\newpage

%-------------------------------------------------------------------------------
\begin{center} Подсчет среднеарифметического \end{center}
\begin{lstlisting}
    def mean_int(numbers):
        res = 0
        for cnum in numbers:
            res += cnum
        return res / len(numbers)
\end{lstlisting}
\newpage

%-------------------------------------------------------------------------------
\begin{center} Нужно добавить поддержку рациональных чисел \end{center}
Рациональное число - пара (числитель, знаменатель). Будем передавать
рациональные числа в виде кортежа.

$$
\frac{a}{b} + \frac{c}{d} = \frac{ad + bc}{bd}\,\,;
\quad
\frac{a}{b} / c = \frac{a}{cd}
$$
\newpage

%-------------------------------------------------------------------------------
\begin{center} Среднее рациональные \end{center}
\begin{lstlisting}
    # rational = (num, denom)
    def mean_rational(numbers):
        num, denom = (0, 1)
        for cnum, cdenom in numbers:
            num = num * cdenom + denom * cnum
            denom *= cdenom            
        return (num, denom * len(numbers))
\end{lstlisting}
\newpage

%-------------------------------------------------------------------------------
\begin{center} Среднее \end{center}
\begin{lstlisting}
    def mean(numbers):
        assert len(numbers) != 0
        if isinstance(numbers[0], (int, long, float)):
            return mean_int(numbers)
        else:
            return mean_rational(numbers)
\end{lstlisting}
\newpage

%-------------------------------------------------------------------------------
\begin{center} Добавляем матрицы \end{center}
\begin{lstlisting}
    def mean(numbers):
        assert len(numbers) != 0
        if isinstance(numbers[0], (int, long, float)):
            return mean_int(numbers)
        elif ismatrix(numbers[0]):
            return mean_matrix(numbers)
        else:
            return mean_rational(numbers)
\end{lstlisting}
\newpage

%-------------------------------------------------------------------------------
\begin{center} "Идея" СА \end{center}
\begin{lstlisting}
    def mean_int(numbers):
        res = 0
        for cnum in numbers:
            res += cnum
        return res / len(numbers)
\end{lstlisting}
\newpage

%-------------------------------------------------------------------------------
\begin{center}  СА \end{center}
\begin{lstlisting}
    add_num = lambda x, y: x + y
    add_rational = lambda (x1, y1), (x2, y2): \
                    (x1 * y2 + y1 * x2, y1 * y2)
    div_num = lambda x, y: x / y
    div_rational = lambda (x, y), z: (x, y * z)

    def mean(numbers, add_func, div_func):
        res = numbers[0]
        for cnum in numbers[1:]:
            res = add_func(res, cnum)
        return div_func(res, len(numbers))

    mean([1, 2, 3] , add_num, div_num)
\end{lstlisting}
\newpage

%-------------------------------------------------------------------------------
\begin{lstlisting}
    def mean(numbers, add_func, div_func):
        it = iter(numbers)
        res = next(it)
        for cnum in it:
            #...
\end{lstlisting}
\newpage

%-------------------------------------------------------------------------------
\begin{lstlisting}
    val = {'num':1, 
           'denom':2, 
           '__add__':add_rational,
           '__div__':div_rational}

    def add_rational(x, y):
        return  (x['num'] * y['denom'] + \
                 x['denom'] * y['num'], 
                 x['denom'] * y['denom'])

    def div_rational(x, y):
        return (x['num'], x['denom'] * y)
\end{lstlisting}
\newpage

%-------------------------------------------------------------------------------
\begin{lstlisting}
    def mk_rational(num, denom):
        return {'num':num, 
                'denom':denom, 
                'add':add_rational, 
                'div':div_rational}
\end{lstlisting}
\newpage

%-------------------------------------------------------------------------------
\begin{center} Нужно автоматически упрощать после операции \end{center}
\begin{lstlisting}
    def add_rational_auto_simpl(x, y):
        res = add_rational(x, y)
        nod = get_NOD(res['num'], res['denom'])
        res['num'] /= nod
        res['denom'] /= nod
        return res

    def mk_rational_auto_simpl(num, denom):
        return dict(num=num, 
                    denom=denom, 
                    __add__=add_rational_auto_simpl, 
                    __div__=div_rational_auto_simpl)
\end{lstlisting}
\newpage

%-------------------------------------------------------------------------------
\begin{center} На предыдущем слайде ошибка \end{center}
\newpage

%-------------------------------------------------------------------------------
\begin{lstlisting}
    def mk_rational(num, denom):
        return {'num':num, 
                'denom':denom, 
                '__class__':Rational}

    Rational = dict(__div__=div_rational,
                    __add__=add_rational,
                    __init__=mk_rational)
\end{lstlisting}
\newpage

%-------------------------------------------------------------------------------
\begin{center} Когда ООП \end{center}
\begin{itemize}
    \item Если есть участок кода, требующий определенного ограниченного 
          набора операций над входными данными
    \item Одновременно в программе могут быть несколько видов подходящих данных, 
          с различной функциональностью для реализации этого интерфейса
    \item Или группировки функций с общим глобальным состоянием
\end{itemize}
\newpage

%-------------------------------------------------------------------------------
\begin{center} Классы не предназначен для \end{center}
\begin{itemize}
    \item Группировки функций
    \item Группировки одной функции
    \item Если вы, ессно, используете нормальный ЯП
\end{itemize}
\newpage

%-------------------------------------------------------------------------------
\begin{center} Проблема \end{center}
\newpage

%-------------------------------------------------------------------------------
\begin{center} Проблема \end{center}
a + b для разных типов
\newpage

%-------------------------------------------------------------------------------
\begin{center} Среднее \end{center}
\begin{lstlisting}
    add = {(int, int):add_int_int, 
           (int, tuple):add_int_tuple,
           (tuple, int):add_tuple_int}

    def register_add(tp1, tp2, func):
        add[(tp1, tp2)] = func

    def mean(numbers):
        res = numbers[0]
        for cnum in numbers[1:]:
            res = add[(type(res), type(cnum))](res, cnum)
        return div[(type(res), int)](res, len(numbers)
\end{lstlisting}
\newpage

%-------------------------------------------------------------------------------
\end{document}
